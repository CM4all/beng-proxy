\documentclass[a4paper,12pt]{article}

\begin{document}
\title{CM4all Beng}
\author{Max Kellermann}

\maketitle

\begin{abstract}
Beng proxy is an HTTP server including an HTTP proxy and a
minimalistic template processor.  Its goal is to dynamically aggregate
a web site from a number of sources (widgets).
\end{abstract}


\section{Features}

Beng-proxy delivers resources via HTTP.  In the most simple form, it
it provides a resource in pass-through mode, acting as an HTTP proxy.

It caches resources if possible.

XML resources can be transformed with XSLT.

On HTML resources, it can apply a simple template language.  This
language provides commands to insert another HTML page, which is
called \textbf{Widget}.

\subsection{Widgets}

A \textbf{Widget} is an object which can be inserted into a web site.
It is rendered by a Widget server into HTML.

We do not assume that we can trust the widget server.  As a
consequence, we have to ensure that a malicious widget server cannot
compromise the security of \emph{beng-proxy}, the client or even other
widget servers.

There is a global registry for well-known preconfigured widgets.  The
user can also choose to run his own (non-registered) widget server.
In fact, any public HTTP server should be able to act as a widget
server.


\subsection{Cookies}

\emph{beng-proxy} can be accessed with cookies switched off.  It includes a
full-featured session management and provides cookies for the widget
servers.

\emph{beng-proxy} maintains the client's session id in either a cookie
or as part of the URI.  In its local session storage, it holds all
cookies which were created by the widget servers.  This way, the
client gets to see only the one session id, disregarding how much
session information may be managed by \emph{beng-proxy}.


\subsection{JavaScript}

Since all widgets are put together into a single HTML page, all of the
JavaScript runs in the same security context.  That will open the door
for malicious widget servers, which are now able to take over the full
web site, including all other widgets.  For that reason, JavaScript is
only allowed for very few well-known and trusted widget servers.  For
all other widget servers, JavaScript use is rejected, or it must be
embedded in an IFRAME, which has technical and practical
disadvantages.


\subsection{Forms}

\emph{beng-proxy} itself does not use the HTTP query string or POST
data.  All of it can be handed off to a widget.  To enable this,
\emph{beng-proxy} rewrites forms, and remembers which widget server to
send the data to.


\section{Widget protocol}

A widget server is simply an HTTP server.  Its content type must be
text/html.  Alternatively, text/xml is allowed, but the caller must
specify an XSLT file for transformation to HTML.


\section{The Beng Template Language}

The \emph{Beng} template language defines commands which may be
inserted in the form of XML processing instructions.

\subsection{Adding a widget}

To add a widget, insert the following command:

\begin{verbatim*}
<?cm4all-beng-widget name="foo"?>
\end{verbatim*}

The following attributes may be specified:

\begin{tabular}{|l|p{8cm}|}
\hline
id & unique identification of this widget; this is required for proper
session and form management if there are several widgets with the same
server URI \\
\hline
name & registered name of the widget server \\
\hline
uri & use an unregistered widget \\
\hline
dock & do not insert the widget here, but at the specified dock \\
\hline
path\_info & optional path info to append to the widget URI; must begin with
a slash \\
\hline
query\_string & optional query string to append to the widget URI
(without the question mark) \\
\hline
\end{tabular}


\subsection{Placeholder for widgets}

When creating a template for multiple pages (called a ``design'' in
the CM-AG slang), you may add placeholders where somebody else might
want to paste a widget.

\begin{verbatim*}
<?cm4all-beng-dock id="foo"?>
\end{verbatim*}


\end{document}
