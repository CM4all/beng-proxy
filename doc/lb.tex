\documentclass[a4paper,12pt]{article}
\usepackage[pdftex,bookmarks]{hyperref}
\usepackage{longtable}

\setlength\LTleft{0pt}
\setlength\LTright{0pt}

\begin{document}
\title{CM4all Beng Load Balancer}
\author{Max Kellermann}

\maketitle

\begin{abstract}
\emph{beng-lb} is a light-weight HTTP load balancer.
\end{abstract}

\setcounter{tocdepth}{2}
\tableofcontents
\newpage

\section{Features}

\emph{beng-lb} listens on a number of socket, and forwards the
requests to another server.  Behind one listener socket, more than one
cluster node may be configured.

The algorithm for selecting a cluster node is configurable: it may be
pure load balancing, pure failover, stickiness with the
\emph{beng-proxy} session cookie, or stickiness with a \emph{beng-lb}
cookie.


\section{Installation}

\emph{beng-lb} requires a current Debian operating system.

Install the package \texttt{cm4all-beng-lb}.  Edit the file
\texttt{/etc/cm4all/beng/lb.conf}.  Finally, restart
\emph{beng-lb}:

\begin{verbatim*}
/etc/init.d/cm4all-beng-lb restart
\end{verbatim*}

\section{Configuration}

Configuration is read from the file
\texttt{/etc/cm4all/beng/lb.conf}.  The following object types can be
configured here:

\begin{itemize}
\item nodes
\item pools
\item listeners
\end{itemize}

\subsection{Nodes}

A ``node'' is one server in a cluster.  It handles requests.  The node
has an IP address, but no port/service.  A node can have multiple
services, it can be member in multiple pools.  Technically, a physical
server can have multiple nodes by having more than one IP address, but
that is only a side effect and is not useful in practice.

Example:

\begin{verbatim*}
node localhost {
  address "127.0.0.1"
}
\end{verbatim*}

\subsection{Pools}

A pool is a cluster of nodes for a specific service.  When a request
needs to be handled, the pool chooses a member and sends the request
to this node.  Upon failure, it may repeat the request with a
different node.

Example:

\begin{verbatim*}
pool demo {
  sticky "failover"
  member "foo:http"
  member "bar:http"
}
\end{verbatim*}

Requests to this pool are always sent to the node named ``foo''.  The
second node ``bar'' is only used when ``foo'' fails (failover).

Other sticky modes:

\begin{itemize}
\item \texttt{none}: simple round-robin
\item \texttt{failover}: the first non-failing node is used
\item \texttt{session\_modulo}: the modulo of the \texttt{beng-proxy}
  session is used to calculate the node
\item \texttt{cookie}: a random cookie is generated, and the node is
  chosen from the cookie that is received from the client
\end{itemize}

Instead of referring to a previously defined node name, you can
configure an IP address instead, and \texttt{beng-lb} creates a new
node implicitly.

\subsection{Listener}

A listener is a socket address (IP and port) which accepts incoming
TCP connections.

\begin{verbatim*}
listener port80 {
  bind "*:80"
  pool "demo"
}
\end{verbatim*}

A listener has a name, an socket address to bind to (including the
port).  To handle requests, it is associated with exactly one pool.

\end{document}
